\documentclass[a4paper,14,4pt]{article}
\usepackage{fancyhdr}
\fancyhf{}
\fancyhead[R]{\thepage}
\renewcommand{\headrulewidth}{1pt}
\renewcommand{\footrulewidth}{1pt}
\fancypagestyle{plain}{}
\pagestyle{fancy}
\usepackage[T1]{fontenc}
\usepackage[utf8x]{inputenc}
\usepackage[croatian]{babel}

\begin{document}


\title{Analiza sistema}
\date{November 2018}

\maketitle
\tableofcontents
\newpage



\section{Cilj projekta}

Razviti veb platformu koja će omogućiti sve neophodne funkcionalnosti za uređivanje elektronskog časopisa. Časopis izlazi dva puta godišnje i tematski je organizovan. Jezik časopisa je engleski.

\section{Učesnici}

    Atributi: id, ime, prezime, korisničko ime, šifra, institucija, email, telefon(opciono), poštanski broj(opciono)
    Korisnici sistema imaju mogućnost upravljanja sopstvenim nalogom (promena korisničkog imena i šifre). Korisnici se međusobno razlikuju po ulogama i privilegijama koje su im dodeljene: Administrator, glavni urednik, urednik, recenzent, autor

    \subsection{Administrator}
    Administrator je korisnik sistema koji upravlja podešavanjima sistema: dodele korisničkih imena i šifara, dodavanje i uklanjanje novih korisnika, upravljanje nalozima, održavanje sistema - dakle stvari tehnološke prirode. On se odmah od početka korišćenja sistema nalazi u bazi podataka - ne registruje se. Iako su mu vidljivi svi podaci o svim korisnicima, radovima i brojevima časopisa, nema pravo dodele uloga (urednika i recenzenata), osim u slučaju dodele uloge glavnom uredniku.

    \subsection{Glavni urednik}
    Na vrhu piramide odlučivanja. Ne registruje se u sistem, administrator ga dodaje. Prilikom prijavljivanja prikazuje mu se u prvom delu prozora spisak novo-prijavljanih radova u sistemu označenih na upadljiv način. U drugom delu rada nalazi se paleta za pretraživanje: po autoru, imenu rada, datumu, statusu rada, statusu urednika, statusu autora.. Odabirom autora prikazuju mu se informacije o autoru. Odabirom rada prelazi se na radni prostor u kojem se ispisuje: naziv rada i informacije o autorima (ime, prezime, status svakoga od njih). Sam rad se nalazi u prilogu. G.urednik ima opciju da: rad odbaci bez konsutovanja sa ostalima i da rad prosledi uredniku. Iz padajuće liste bira urednika kojem šalje rad, u suprotnom, rad se arhivira kao odbačen. Glavni urednik ima pravo i da dodeli recenzente, urednike i da odluči koji će se rad objavljivati u kojem broju, da doda autora na crnu listu. Ima pristup odeljku "Upravljanje korisnicima" u kojem može da dodeli uloge urednicima/recenzentima, ili da ih oduzme. Ako želi da obriše nekog korisnika iz sistema, mora da pošalje poruku administratoru sistema. Takođe ima mogućnost ostavljanja komentara na rad.

    \subsection{Urednici}
    Urednici su specijalizovani za određene oblasti (oblast se pridodaje kao atribut). Primaju od g.urednika rad, čitaju ga i šalju predloge recenzentima. Urednik ima pravo da odbaci/prihvati rad, iako recenzenti misle da je rad za objavljivanje, ali onda treba da napiše obrazloženje. Ako urednik misli da je radu potrebna neka izmena, ima opciju da rad označi za menjanje, označava ga i otvara mu se polje gde unosi komentar urednika, koji objašnjava šta je sve potrebno promeniti. Ako komentar  na rad već postoji, on se učita u prostor za pisanje komentara i urednik ga menja ili briše i dodaje novi. Komentar urednika je opciona stavka. Po istom principu funkcionište i odbijanje/prihvatanje bez recenzije. Urednik prilikom logovanja na sistem ima u gornjem delu prozora spisak novih radova koje mu je poslao g.urednik. U donjem delu prozora je paleta za pretraživanje, opisana kao kod g.urednika. Odabirom na novi rad urednik ga ili odbaci, ili mu dodeli recenzente.

    \subsection{Recenzenti}
    Prilikom logovanja imaju spisak novih radova poslatih od urednika. Rad mogu da odbiju, i dalje nemaju više nikakva zaduženja vezana za taj rad. Takođe, mogu da pristupe i spisku svih dosadašnjih radova: prihvaćenih i odbijenih. Nakon pročitanog rada, recenzent popunjava formular za recenziju i objavljuje je. Potencijalni recenzent je onaj koji: je već recenzirao neki rad, ko je  kao korisnk čekirao polje "nemam ništa protiv da me kontaktirate za recenziranje nekog rada"

    \subsection{Autori}
    Dodatni atributi: broj dosadašnjih prilaganja radova, broj dosada objavljenih radova. Postoji i crna lista nepoželjnih autora koja sadrži ime autora i razlog zbog kog se nalazi na listi:  plagiranje, različiti vidovi varanja utvrđeni od strane ostalih korisnika sistema. Autori prijavljuju rad. Prilikom prijave ispunjavaju formular i označavaju da li je to novi rad, ili nova verzija rada koji je označen za ispravku. Ako u nekom trenutku žele da povuku rad, šalju zahtev za povlačenje rada.

\section{Časopis}
Predstavlja jedno izdanje časopisa i opisan je atributima: ISSN, godina, naslov, g.urednik, urednici, minimalan i maksimalan broj radova po izdanju.

\section{Rad}
    Atributi: broj, naslov, godina, imena autora, datum prve prijave, datum poslednje prijave, log prijava, rad u pdf formatu, recenzenti, urednik, status, broj verzija, objavljen

    Statusi rada:
    \begin{itemize}
        \item prijavljen: dobija status kada ga autor prijavi
        \item na recenziji: kada ga primi recenzent na recenziju
        \item na doradi: ovaj status dobija kada ga g.urednik/urednik označi da treba da ide na doradu
        \item povučen: ako je autor podneo zahtev za povlačenje rada
        \item odbijen bez recenzije: od strane g.urednika/urednika. U tom slučaju je potrebno priložiti komentar (funkcioniše isto kao kada se rad označava za doradu). Rad je odbijen bez recenzije i ako urednik nije mogao da pronađe recenzente koji žele da recenziraju rad, a sam nije želeo da čita ceo rad
        \item odbijen sa recenzijom
        \item prihvaćen bez recenzije: slično kao odbijen bez recenzije, potrebno je ostaviti komentar
        \item prihvaćen sa recenzijom
    \end{itemize}


\section{Prijavljivanje i registrovanje korisnika}
Prilikom registrovanja, korisnik ima mogućnost da označi opciju "nemam ništa protiv ako želite da me odaberete kao recenzenta". Ako ovu opciju nije čekirao, može to da učini kasnije kada se prijavi.
\begin{itemize}
    \item Registrovanje: popuni se formular i pošalju podaci. Administrator ih prima i šalje šablon1 u kojem korisnika obaveštava koje mu je korisničko ime i šifra
    \item Prijavljivanje: korisnik popuni formular i time se uloguje
\end{itemize}

\section{Komunikacija među korisnicima}
Korisnik odabere šablon, otvori mu se radna površina za pisanje u koju se šablon učita. On taj šablon može sačuvati i ima opciju da ga pošalje, čime se šablon automatski šalje na mejl drugog saučesnika u komunikacji. Korisnik može i napraviti novi šablon i obrisati ga, ako nije predefinisan od strane sistema, kao što su dole navedeni šabloni.
\begin{itemize}
\item Administrator ostalim korisnicima: šabloni 1 (uspešno registrovanje) i 2 (uspešna/neuspešna promena ličnih podataka)
\item G.urednik urednicima: šablon 3 (o novopristiglom radu)
\item G.urednik administratoru: šablon 4 (o brisanju korisnika iz sistema)
\item Urednik recezentima: šablon 5 (predlog rada za recenziranje)
\item Urednik autoru: šablon 6 (o tome kako je rad prihvaćen i šta treba da priloži) i  šablon 7 (odbijen i obrazloženje)
\item Recenzent uredniku: šablon 7 (o tome da li odbija ili prihvata recenziju)
\end{itemize}

\section{Životni ciklus rada}
\subsection{Prijavljivanje}
Rad prijavljuje autor koji je ujedno i "odgovorno lice" za taj rad. Spisak autora koje rad sadrži kao atribut mogu biti reference na autore već poznate sistemu, inače o autoru će biti napravljen zapis u sistemu koji će biti moguće koristiti kao njegov kontakt u slučaju da je to potrebno (npr. poziv za učlanjenje).
\subsection{Dodeljivanje urednicima i recenzentima}
Rad je inicijalno prikazan glavnom uredniku koji ga dodeljuje uredniku. On zatim "nudi" rad recenzentima koji imaju mogućnost da prihvate ili odbiju tu ulogu. Obaveštenja o dodelama radova i ponudama recenzentskih uloga su realizovana putem slanja šablona (opisanim u delu 6) i praćena su akcijama sistema.
\subsection{Recenziranje}
Recenzent može prihvatiti rad koji mu je ponuđen na recenziju od strane urednika nakon čega automatski biva dodeljen tom radu. Od njega se očekuje da popuni formular koje će sadržati delove koji su upućeni autoru/autorima i uredniku. Nakon popunjavanja objavljuje svoju recenziju koja će biti vidljiva uredniku, koji preuzima dalje akcije.
\subsection{Komentarisanje rada od strane urednika}
Rad može biti komentarisan od strane glavnog urednika i urednika i to u slučajevima kada je rad odbijen sa komentarom, koji će autor moći da pročita kao objašnjenje za preduzetu akciju, ili kada se rad označava za doradu, čime se autoru stavlja do znanja koje su to promene koje se zahtevaju.
\subsection{Ažuriranje}
Nakon što je rad od strane urednika označen za doradu, autor ima mogućnost da izmenjenu verziju rada ponovo prijavi. U tom slučaju potrebna je ponoviti i dodelu recenzenata. Kao u slučaju prvobitne dodele, uredniku je ponuđen spisak recenzenata na kome bi favorizovani bili oni koji su na tom radu već imali recenzentsku ulogu.
\subsection{Menjanje statusa}
Status rada je inicijalno postavljen na "prijavljen" i može biti promenjen akcijama korisnika sistema. Moguća stanja i akcije koje dovode do tih stanja su navedene u delu 4.
%\subsection{Prijavljivanje rada kao plagiran/uvredljiv sadržaj}
%Ovo pravo imaju samo registrovani korisnici.

\section{Upravljanje časopisom}
Upravljanje časopisom je uloga glavnog urednika i obuhvata poslove postavljanja imena časopisa, minimalan i maksimalan broj radova, kao i izbor samih radova koji će u njemu biti objavljeni.
\section{Sistemska podešavanja}
Obavlja ih administrator: upravljnje podacima u bazi, dodavanje novih korisnika i uklanjanje postojećih... Na nivou sistema treba da postoji log svih aktivnosti koji treba da bude vidljv administratoru i glavnom uredniku

\section{Šabloni i formulari}
Sadržaj šablona i formulara biće priložen nezavisno od ovog dokumenta u odeljku "Prilozi (Attachments)"

\end{document}
